\documentclass[12pt,a4]{article}
\usepackage[latin1]{inputenc}
\usepackage[T1]{fontenc}
\usepackage[letterpaper,colorlinks,breaklinks,backref]{hyperref}

\newcommand{\metax}[1]{meta${}^{#1}$}
\newcommand{\metan}{\metax{n}}

\begin{document}
\title{Notes on solving the Bedlam cube}
\author{Far� \copyright 2001}

\maketitle

\section{The problem}

\subsection{The Cube}

The problem consists in building
a $4 \times 4 \times 4$ 3D cube from 13 pieces.
The shape of the pieces are given, with 12 pieces being made of 5 little cubes,
and one piece being made of only 4 little cubes.

Although originally submitted to me as a "soma cube" by Rik Rose
\url{http://rikrose.net/soma/},
the problem really is a Bedlam cube \url{http://www.bedlamcube.com/}
as invented by Bruce Bedlam,
and there is already at least one web page
\url{http://ch.twi.tudelft.nl/~sidney/puzzles/}
where a solution for it is given -- or rather,
all the $19,186$ solutions.

Note: the original $3 \times 3 \times 3$ soma cube was invented Piet Heine,
\url{http://home8.inet.tele.dk/bundgard/SOMA/SOMA.HTM} and the name was
indeed chosen from the name of the perfect drug in Aldous Huxley's
``Brave New World'' \url{http://www.huxley.net/bnw/}

\subsection{The Challenge}

Either one solution is wanted, as fast as possible,
or all solutions are wanted, as fast as possible.
Solutions should not usually include similar configurations obtained
by rotating the cube, or rotating a piece in a way that preserves
the identities of the little cubes it fills.
I've decided that a reasonable interpretation of the problem
was to find all solutions, up to symmetry
(without repeating symmetric versions of a same solution),
as fast as possible.
Solutions can be printed as listing of which piece is placed where
(rotation and translation involved), or as a cubic array mapping
which little cube is filled by which piece.

An ``optimized'' program would just take the list of solutions and dump it.
Why is it considered as cheating? Because it requires prior knowledge of
the solutions. So what? Especially when we're considering an input-less
deterministic program: it's a mathematical problem the solution is immanent
in the universe. An approach to pointing down the difference could be
one of Kolmogorov complexity: considering the intrinsic complexity
for a human brain to solve the problem, there is no way it can be solved
but by some external artefact (computer program, godly message, etc.),
so we can trace any correct reply to such artefact, and consider as a
valid a reply by the author of such artefact. Then we can recurse the
problem by questioning what's an author, but let's not do it here and now.

\subsection{Annoying Complication}

Should only be considered acceptable solutions that can be
physically constructed by adding pieces one after the other.
I think the simplicity of the actual pieces of this puzzle
makes it the case that all solutions are constructible.

However, until a proof or disproof of such conjecture is given,
or if we are to write a program that work in more situations,
we need to test the following:

\begin{itemize}
\item
 In a combination of pieces, we call reducible the pieces that
 can be removed by just pulling them in one of the 6 directions:
 there is no obstacle to them in that direction
\item
 If by removing reducible pieces, and starting again,
 we can reduce the combination to an empty one,
 then the solution was acceptable;
 if not, print specially the reduced irreducible combination
 (that could themselves be split in several groups
 until irreducible groups are reached).
\end{itemize}

\section{The Solver}

\subsection{My Approach}

I didn't have any practical experience with CL before I wrote this;
the very purpose of this was as an exercise to be keener in CL.
At first, I've tried to write things in such a way as to keep a generic parts
as a reusable skeleton that can be copied and modified.
Later on, I tried to bum the program.

\subsection{What I Originally Did}

Early on, I decided to breaking the symmetry of the search,
which was a predictable win of over an order of magnitude in search time
at the expense of some non-negligeable work at development time.
If you consider that multiple isometric solutions must not be output,
then it was a good thing; if you're only interested in getting one solution,
or all, even isometric ones, or in not wasting too much time,
then it was a bad move; otherwise, it was a good thing.
Retrospectively, it's a great thing for an automated system to do,
or for a semi-automated system to allow to specify declaratively,
but not forcibly worth doing manually.

Early on, I decided to precompute rotations, translations, etc.,
which is also a good thing, and should be doable by simple declaration.
Some of these precomputations happened to be more expensive.

I spent a lot of time trying to bum the search speed for the initial
stupid search strategy (for each cube, try to fit a piece in it), then
a slightly improved one (for each piece, try all precomputed variants).

Later on did I opt for forward constraint propagation:
for each shape, restrict further sets of shapes.
Because the encoding of constraint propagation made it natural,
I added static strategy choice of pieces to insert.

Since this still was not enough, I added dynamic strategic choice.
I also added some program profiling,
and concluded my current search was 1.e10 times too slow
(1.e8 in optimized compilation),
which clearly indicates that the algorithm's global search tree is too large,
notwithstanding  any low-level optimization.
Doing this earlier could have saved a lot of time.

Nevertheless, after some forty minutes
(out of an estimated approximate total of 6 months to 1 year),
three solutions were found. This reduced my interest in soma.

I nevertheless added sequential search restart capability,
so that incremental efforts in optimization
supposing a same search tree could help cover the search space:
either low-level optimization or parallelization.

After some ten hours or so of cumulated search
(with minor implementation tweaks, including bug-fixes,
and restarting from previous checkpoint),
exploring an estimated 3.e-3 of the search space
(6.5e7 attempts out of an estimated 2.e11 to go)
yielded 364 solutions, out of an estimated total of 1.e5 solutions.

I considered my search algorithm as dumb, but as good as I could do,
and intended to use various low-level techniques
(removing CONSing, boxing of bitmasks, and parallelizing)
to bring the whole search thing down to a few hours.

\subsection{A better Algorithm}

After kind of solving the problem, albeit in an unsatisfactory way,
I asked to see the first solution submitted to Rik Rose,
that was written in C by ``some guy from irc.bsdunix.net''.

Although this guy used none of the representation optimizations I used,
and did not do forward constraint propagation, he was thousands of time
faster than I: because he had found a fundamentally better search algorithm,
in a way that I had failed to see, which demonstrates that the author
had a deep understanding of the problem, and that despite using a low-level
language, he hadn't been lured into focusing on irrelevant low-level points.

His search solution uses a non-trivial property of the problem
to simplify its search, that I hadn't used up to then,
and reduces his search considerably,
by eliminating the need to consider arbitrary translations.
I'm ashamed that I almost found that trick myself, but that I missed it
completely after I precomputed translations and went into narrowing bitmasks,
which hid this important geometric property, that
the lexicographic order of the bitmap is preserved by translation:
i.e. if a shape made of little cubes is translated,
the relative lexicographic order of these cubes is preserved.
Thus, if all previous cubes (in the lexicographic order) have been filled,
the only cube that can fill the current one is the lexicographically
smallest in the shape of the translated rotated piece.
Hence, the right search strategy is to take each cube as a spot in order,
and try fill it with the least cube of every possible shape
of every possible piece. It is still possible to use all my optimizations,
yet this one optimization beat all of mine --- choosing the right search tree.

\section{Analysis Of My Attempts}

\subsection{My failed attempts}

I wasted lots of time not finding a proper solution.

My mistakes, from the most superficial to the deepest, were
\begin{itemize}
\item
lots of typos that a proper development environment could have easily
detected, and that CMUCL was much slower to report than, e.g. OCAML would,
and that a better development environment could have avoided altogether
by a closer interaction between edition and semantic interpretation.
\item
failure to propagate constraints correctly when transforming the program,
that a static type-checker could have detected, and that that a dynamic
static typing environment could have mostly propagated automatically.
\item
focusing on low-level representational problems,
because I distrusted the speed of my implementation (which indeed
sucked at times, and did not provide useful feedback on time usage),
instead of focusing on better algorithmic choices.
-- I clearly need detox from C thinking habits.
\end{itemize}

\subsection{Other optimization paths to consider}

\begin{itemize}
\item
LISP compilers don't support them, and I'm told 64-bit integers
 are very slow with GCC, anyway, so use 32-bit integers instead.
\item
Instead of hand-translating to C, either use ThinLisp,
 or translate to Scheme and use Stalin.
\item
when a solution is found, see if it can be reduced in groups of pieces
 some non-trivially groups of which are globally stable by symmetry.
 This allows to simplify the presentation and explore the structure
 of the solution set. However, it is yet unclear to me if it could help
 eliminate part of the search tree: dynamically introducing new 'metapieces'
 could accelerate the initial finding of more solutions, but might actually
 slow down the overall search, since most of the time is spent eliminating
 bad combinations, not finding good ones.
\item
more interesting would be to have some better analysis of the geometric
 interactions between pieces than provides the current stubborn bitmap.
 A clever such analysis, if possible, could provide a many-fold speed
 improvement in search speed by detecting dead sub-trees early on, in
 combination with an appropriate search strategy.

\item
It is ugly because of all the bumming tricks it uses,
 that make it quite difficult to change the structure of the program
 or reuse parts of it. Which is not a problem at all in such a small
 and simple program, where no sharing, extension, etc., is ever needed.
\item
For instance, the use of ad-hoc single-linked structures with manual
 management makes perfect sense in presence of linear structures,
 but is unusable in programs with sharing.
 In LISP, you don't have that and cannot have that, because your program
 uses general higher-level tools, that your compiler cannot specialize
 because of the ability to dynamically modify things (a superoptimizer
 for static programs could, but then, it's too late, because you don't
 design your programs with that in mind).
 I suppose a compiler for Clean, however, could optimize all those things,
 thanks to its linear type-system that can detect and automatically use
 these optimization opportunities, while keeping the program in a nice
 high-level style that allows to easily make high-level changes to it,
 and to detect mistakes in breaking linearity invariants.
 I'm sure a solution in Clean would be the shortest, fastest and cleanest
 of all -- PLUS, the Clean implementation can already and readily
 take advantage of concurrence!

\end{itemize}


Optimizations intended after reading this solution:
\begin{itemize}
\item
First, and foremost, do use the same search tree as that guy!
\item
 Keep doing forward constraint propagation, keeping for each live piece
  and all live spots (sorted in opposite order) the list of live positions
  of this piece where this spot is the smallest-ranked.
\item
 It might pay lot to be big-endian in such matters, so as to sort shapes
  trivially in row-major order. Also, an bummed loops will only handle
  the lowest 32 bits past the first half of the tree. Similar bumming
  could special case skipping over the test for shapes with high low-spot
  until needed, but since this complication would only help at the first
  steps of the search, it's a lot of complication for little. If there
  had been are lots of significant inhomogeneity in treatment, shapes
  could have be represented as a vector of bummed functions that do the
  right thing.
\end{itemize}

Expected gains from various other optimization (log10-scale speed improvement)
\begin{itemize}
\item
+1.6 from using a better algorithm
\item
+0.6 from removing consing altogether,
  allocating search lists in a proper arena
\item
+0.7 by using (unsigned-byte 32) everywhere instead of fixnums and bignums,
 and doing away with typechecking altogether.
\item
+0.1 from translating to C with ThinLisp or Stalin, doing away with
 any remnants of low-level inefficiency AND allowing to distribute
 computations over many machines...
\item
+0.6 from parallelization on many machines available to me
 (dumping some 3.e5-odd specialized C programs to be scheduled
 on a wide distributed system, by specializing after 3 levels of search)
These could bring the whole tree search into human-scaled times,
a few hours.
\end{itemize}

From my many attemps, some lessons can be drawn about
a high-level development environment for a declarative language,
that could help automate my implementation efforts:
\begin{itemize}
\item
The choice of data-structure representation can be automated
 by considering the actual operation pattern on which to amortize,
 or for which to provide guarantees.
\item
 for sets, I used lists, simple or sorted, bit-vectors
  (that sucked big-time under CMUCL when trying to sort them),
  bignums-as-bitmaps (C could use 64-bit integers directly).
  Considering mostly total map/reduce access pattern,
  I didn't use hashes or trees.
\item
I transformed the evolving list of the *solution-path* (now
 *search-path-FOO*) into a constant vector only a partial filling of which
 is meaningful -- that is, I manually gave a preallocated arena
 for an object with predictable allocation pattern,
 changing the low-level access protocol at the same time.
\item
The above transformation worked because the occasional sharing of
 the *solution-path* in *solutions* was only interesting up to copy,
 and the rare copy was very affordable, so sharing could be done without.
\item
In optimized brute-force searches, where every bit of efficiency counts,
 and an accumulated object is changed incrementally, it is sometimes good,
 sometimes bad, to recover the object by decremental change rather than by
 save and restore. I did this for the cube when it was still a large array,
 rather than a bit vector.
\item
I changed the cube from an array of piece ids to a bitmask,
 keeping the ids only in the solution path, since they were not used
 in the critical search path, only at success time.
\item
Because it induced pervasive changes everywhere, I didn't try
 changing datastructure representation to accomodate with index optimization:
 restructure arrays and sequences so that index-intensive search be most
 effective -- stubbornly decrementing from max to 0, instead of going
 through extra indirections and checking.
\item
More generally, it's a PITA to maintain manually several different
 low-level representations of the 'same' object at different times,
 because programming languages mostly insist on having a one canonical
 representation for every object.
\item
Another room for improvement is in cacheing expensive precomputations
 across modifications of the program, which an automated help should prove
 is correct by tracking dependencies (providing options to keep marked
 versions, compare them to new ones, etc.).
 LISP allows easy read/write of things, as long as they do not
 contain code per se (which can be worked around by metaprogramming).
\item
Some functions had to be rewritten because CMUCL couldn't optimize
 them for the particular cases in which they were used, being stopped
 by higher-order calls to dynamic functions, whereas these dynamic
 functions were actually static (LISP has no way of specifying that)
 and very much specializable.
\item
When translating to C, we can do away with dynamic allocation,
 and even with the stack, using special arenas for everything.
 The stack is but a particular arena, although a special strategy
 might be to use implicit stack allocation in some cases.
\item
When the search tree is well-defined, deterministic, etc.,
 it can be automatically parallelized and massively distributed.
\item
in case where there is not total trust in distributed search,
 redundant clients can be run, and a table can be maintained (or not)
 to memoize schedules and/or results (if they cannot be deduced simply
 from the schedule and unified table -- or maybe rather the unified
 table can be computed from the schedule -- or maybe the implementation
 maintains several versions)
\item
the first few levels are handled by a small parallel hierarchy
 of lisp programs on trusted servers. When they have sufficiently
 reduced the search space, specialized C programs are dumped for
 the searching the current sub-tree.
\item
''sufficiently'' is determined according to some dynamically-checked
 heuristics to balance C and specialized code speed versus
 C compilation and specialization slowness),
\item
Since determinism is useful for comparison, these dynamic heuristic
 evaluations for speed should use pseudo-randomized samples with
 a good-quality portable deterministic PRNG and a shared memoized random seed.
\item
metaprogramming allows to maintain reliably several different coherent
 versions of a 'same' structure, each corresponding to a different aspect
 of it (valid at different times, maybe interleaved), whereas it's a hell
 to do it consistently (let alone efficiently or maintainably) by hand.
\item
specialized C programs for searching subtrees can be queued and dispatched
 to less trusted and non-LISP-running machines.
\item
Some of the C infrastructure needs be compiled but once per architecture;
 other data needs be compiled for every subtree.
\end{itemize}

\subsection{Digging meta-levels}

descending and digging are verbs I use for exploring and
 actually making meta-levels suitable for metaprogramming,
 knowing that such activities extend implementations DOWNWARD, not UPWARD.
\begin{itemize}
\item
 Some may prefer to see meta-level exploration as climbing,
  considering that they go to higher, more unreachable places,
  whence information flows downwards.
  But I feel that digging large surfaces down to hell to provide deeper and
  deeper foundations to the computations you build gives a more accurate vision
  of what's involved in metaprogramming than that of making an occasional
  tourist excursion in a high peek.
\item
 Plus there's virtually no limit to the amount and variety of digging
  efforts, whereas you go to high places takes only so much effort in so
  many different ways, and you can only provide so much infrastructure
  once up there.
\item
 Finally, the analogy of digging so as to program downward levels
  usefully extends in reminding us that the taller the final building
  we aim at achieving, the deeper and wider the foundations we need.
\item
 So yes, meta-level software serves to go higher up on top of it;
  but building that meta-level software really means going down.
\end{itemize}

So if we consider a complete computing system as a building,
 and go down from the top of computing systems, here is what we see,
 each level either being manually entered by some user or \metan-programmer,
 or being the topmost result of automatic computations by underlying
 infrastructure at below levels, themselves subject to similar criteria.
\begin{itemize}
\item
 at the top, there is raw final data, as manipulated by the user
\item
 one level below, there are programs, as written by the programmer.
\item
 then, there are meta-programs, like macros, or compilers, that
  transform programs into other programs.
\item
 then, there are meta-meta-programs, that manipulate meta-programs,
  and choose the right combination thereof.
\item
 etc.
\end{itemize}

We can see how our optimization process could descend meta-levels:
\begin{itemize}
\item
 at the top, data, either input by us or output by the program,
  as final or intermediate results. Modifying these data is making
  computation by hand -- doing the work of a computer.
\item
 below, the many versions and components of our search program,
  as processed by the system and usually entered by the programmer.
  Modifying these programs by hand, so as to transform and optimize them
  is making (meta-level) computation by hand -- doing the work of a computer.
\item
 then, there are meta-programs that help build programs. They can apply
  some optimization, automatically maintain coherence between various
  internal and external encodings, etc. Having them is great, but once again,
  explaining which meta-program to use when and to combine how is some work
  that could be automated.
\item
 a program that would search for proper tactics to combine meta-programs,
  based on analyses of programs is a meta-meta-program.
  Making these tactics reversible and redoing these analyses dynamically
  can provide for quite a sophisticated meta-meta-program that could
  adapt its tactics at runtime to the actual running data and
  speed of programs. Having them is great, but etc.
  In dynamic setting, using some generic meta-program to make transformations
  incremental could also be usefully used within such meta-meta-programs
  or on them.
\item
 a program that would transform meta-meta-programs so as to learn from
  dynamically acquired information on the speed of programs and the
  return of investment from various optimizations would be a {\metax 3} program.
\end{itemize}

\subsection{Touching the ground}
\begin{itemize}
\item
  somewhere, deep below, is a system hardware and the compiler that usually
  running at some level above the raw hardware. Separating the user and
  this hardware, arbitrary levels of \metan-programs might be involved.
\item
  it is very important that at every stage of the development, the system
  "touch the ground" -- that is, is actually executable. Digging too deep
  at once, or building roofs when the walls are not there yet, means that
  a stage will be useless even if complete, until other stages are ready,
  and even then, might require adapting code to fit walls as they finally
  are rather than as they were designed when coding begun.
\item
  In other words, the system should be worked in incremental steps, each
  of which leaves a working system when done. When scheduling modifications
  to work on, modifications that depend on further modifications so as to
  be usable should be avoided (unless, at some metalevel, they are useful
  for designing the further component, even though they are not for
  implementing it).
\end{itemize}


(I suppose I could avoid side-effects in macros altogether by ugly hacks
involving putting everything in a huge macro that propagates additional
information monadically -- but yuck)

\end{document}

;; This file depends on the library file fare.lisp
;; See notes in soma-notes.txt

"
[rik] i've only got a C solution, and that can afford to just literally
      brute-force through every position and rotation of all the pieces
[rik] it takes quarter of a second, on a p2-300
>rik< what format do you expect for output?
[rik] <shrug> anything, as long as it's understandable in some form
[rik] the first one i had was printing the piece numbers in a 4 by 4 grid,
      with 4 grids next to each other, the leftmost grid being \"top\", and the
      rightmost being \"bottom\" layers
[rik] the other way is to say where in the cube the pieces are
[rik] i don't mind which. either one is equally hard on my poor little brain :)
>rik< do you want solutions up to symmetry, or including symmetric ones?
[rik] the program can stop when it finds its first solution. I'm not looking
      for all of the possible solutions
"

;; For the original 3x3x3 soma by Piet Heine, see also
;;      http://home8.inet.tele.dk/bundgard/SOMA/SOMA.HTM
;; The name comes from the drug in Aldous Huxley's 'Brave New World'
;;      http://www.huxley.net/bnw/
